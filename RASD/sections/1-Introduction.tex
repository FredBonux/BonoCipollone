\chapter{Introduction}
\section{Purpose}
The purpose of this document is to analyze and define the goals and requirements to later design, on behalf of the e-Mobility Service Providers (eMSPs), the infrastructure for the eMall App.

\subsection{Goals}
\begin{tabular}{ |l|l| } 
 \hline
G1 & Allow users to find charging station nearby and their tariffs\\
 \hline
G2 & Allow users to book a charge session in one of the Charging Point\\
 \hline
G3 & Allow users to authenticate to the Charge Point and start the charging session\\
 \hline
G4 & Allow users to check the status of an active charging session\\
 \hline
G5 & Allow users to be notified when the charging process is completed\\
 \hline
G6 & Allow users to pay for the charging session\\
 \hline
\end{tabular}

\section{Scope}
Electric mobility (e-Mobility) is a way to limit the carbon footprint caused by our urban and sub-urban mobility needs. When using an electric vehicle, knowing where to charge the vehicle and carefully planning the charging process in such a way that it introduces minimal interference and constraints on our daily schedule is of paramount \\\\
eMall App is a platform that helps the end users to plan the charging process, by getting information about Charging Points nearby, their costs and any special offer; book a charge in a specific station, control the charging process and get notified when the charge is completed. It also handles payments for the service.\\\\
In the e-Charging ecosystem, there are many different actors involved that we need to keep into consideration while collecting requirements and designing the system. The first information to consider is that Charging Points are owned and managed by Charging Point Operators (CPOs) and each CPO has its own IT infrastructure, managed via a Charge Point Management System (CPMS). \\\\
In order to comunicate with each actor, the 
\href{../Specs/OCPI-2.2.1.pdf}{OCPI (Open Charge Point Interface) protocol} is used.

\subsection{World Phenomena}
As this system will act as a middleman between the users and the various Charging Point CPMSs, we thought that splitting the phenomena between "user side" and "CPMS side" can help to better understand.

\subsubsection{User side}
\begin{tabular}{|l|l|}
	\hline
	WP1 & User decides to charge the electric vehicle\\
	\hline
	WP2 & User goes to the Charging Point\\
	\hline
	WP3 & User connects the Electric vehicle to the Charging Slot\\
	\hline
	WP4 & User disconnects the Electric Vehicle from the Charging Slot\\
	\hline
\end{tabular}

\subsubsection{CPMS side}
\begin{tabular}{|l|l|}
	\hline
	WP5 & Charging Point starts to provide energy to the Electric Vehicle of the User\\
	\hline
	WP6 & Charging Point ends to provide energy to the Electric Vehicle of the User\\
	\hline
	WP7 & CPO starts the maintenance of a Charging Slot\\
	\hline
	WP8 & CPO completes the maintenance of a Charging Slot\\
	\hline
\end{tabular}

\subsubsection{Payment Service Provider side}
\begin{tabular}{|l|l|}
	\hline
	WP9 & Payment provider charges successfully the payment method registered by the user\\
	\hline
	WP10 & Payment provider fails to charge the payment method registered by the user\\
	\hline
\end{tabular}\\\\

\subsection{Shared Phenomena}
\subsubsection{User side}
\begin{tabular}{|l|l|}
	\hline
	SP1 & User registers an account\\
	\hline
	SP2 & User verifies the email for his account\\
	\hline
	SP3 & User add payments information for his account\\
	\hline
	SP4 & System shows the nearby available charging points to the user\\
	\hline
	SP5 & User books a charging session through the system\\
	\hline
	SP6 & System send information about a charging session to the User\\
	\hline
	SP7 & User starts the charging session through the system\\
	\hline
\end{tabular}

\subsubsection{CPMS side}
\begin{tabular}{|l|l|}
	\hline
	SP8 & System books the charging session for the user via the CPMS\\
	\hline
	SP9 & CPMS send to the System the charging session details\\
	\hline
	SP10 & System authenticate the charging session for the CPMS\\
	\hline
\end{tabular}

\subsubsection{Payment Service Provider side}
\begin{tabular}{|l|l|}
	\hline
	SP11 & System send cost information to the Payment Provider to charge the User\\
	\hline
	SP12 & Payment Provider send to the system the payment process details (eg. status)\\
	\hline
\end{tabular}\\\\

\section{Definitions, Acronyms, Abbreviations}

\subsection{Definitions}

\begin{tabular}{|p{5cm}|p{10cm}|}
	\hline
	Charging plug & Physical connector that allow to transfer energy to the connected vehicle\\
	\hline
	Charging slot & Physical device with multiple Charging plugs that can charge electric vehicles\\
	\hline
	Charging Point & Physical structure composed by multiple Charging Slots\\
	\hline
	Maintenance of a charging slot & Activity/activities that results in a momentary unavailability of the charging slot\\
	\hline
	Payment information & information required by the payment provider to be able to charge the user for the service (e.g. credit card number)\\
	\hline
	Charging session & period of time when the vehicle is connected to a charging plug for charging\\
	\hline
	Booking period & period of time between the booking of a charging session and the beginning of the charging session\\
	\hline
	{Guest \newline Guest User} & Unregistered user\\
	\hline
	{User \newline Enabled User \newline Active User} & Registered user with confirmed email and payment method\\
	\hline
	Unconfirmed User & Registered user without confirmed email\\
	\hline
	Pending User & Registered user with confirmed email but no payment method set up\\
	\hline
	Payment Service Provider & External service that provides API to process payments\\
	\hline
\end{tabular}

\subsection{Acronyms}
\begin{tabular}{|l|l|}
	\hline
	eMSP & e-Mobility Service Provider\\
	\hline
	CP & Charge Point / Charging Point\\
	\hline
	CPO & Charging Point Operator\\
	\hline
	CPMS & Charging Point Management System\\
	\hline
	OCPI & Open Charge Point Interface\\
	\hline
	EV & Electric Vehicle\\
	\hline
\end{tabular}

\section{Revision History}
\begin{itemize}
	\item v1.0 - 21 December 2022
\end{itemize}

\section{Related Documents}
\href{../Specs/OCPI-2.2.1.pdf}{OCPI specifications document (OCPI-2.2.1.pdf)}

\section{Document structure}
The document is structured in six sections:

\begin{enumerate}
	\item Problem introduction, associated goals of the project. In this section you can also find the scope of the project, the various phenomena occurring and the definitions and abbreviations used in this document.
	\item The second section contains a overview of the systems, the details about the users types and the main functions. Class diagrams, statecharts and domain assumptions are used to introduce the various scenarios.
	\item In this section are specified the different requirements of the system, the various use cases and the mapping between functional requirements and the goals of the system.
	\item Alloy is used to conduct the formal analysis of the system.
	\item Total effort
	\item References used
\end{enumerate}






















