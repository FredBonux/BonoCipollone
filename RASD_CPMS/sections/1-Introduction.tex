\chapter{Introduction}
\section{Purpose}
The purpose of this document is to analyze and define the goals and requirements of the Charging Point Management System (CPMS), the IT structure capable of managing a car charging station.
\subsection{Goals}
\begin{tabular}{ |l|l| } 
\hline 
G1 & Allow eMSP to start and end a charging session\\
 \hline 
G2 & Allow eMSP to be notified when the charging process is completed\\
 \hline
G3 & Allow CPO to have information about the internal status of a charging station\\
 \hline
G4 & Allow eMSPs to have information about the current tariffs\\
 \hline
G5 & Allow eMSPs to have information about the position of the charging stations\\
 \hline
G6 & Allow eMSPs to have information about the status of the car charging process \\
 \hline
G7 & Allow CPOs to manually decide the current energy price \\
\hline
G8 & Allow CPOs to manually decide from witch DSO acquire energy \\
 \hline
G9 & Allow CPOs to manually decide whether to use or store energy inside batteries \\
\hline
\end{tabular}

\section{Scope}
Each charging station managed by the Charging Point Operators (CPO) has an IT infrastructure called Charge Point Management System (CPMS). Through this application, administrators are able to manage the various charging stations such as the selection of the DSO and the use of the batteries present in the stations. The CPMS must be able to manage the station,deciding independently the most optimal energy management policy. The CPMS is able to connect the charging station with the rest of the e-Charging ecosystem, allowing an interaction with the user and with the various energy suppliers (Distribution System Operators (DSO)).\\

\subsection{World Phenomena}

\subsubsection{eMSP side}
\begin{tabular}{|l|l|}
	\hline
	WP1 & eMSP wants to start a charging session\\
	\hline
	WP2 & eMSP wants to know the rates offered by the various CPMS\\
	\hline
	WP3 & eMSP wants to book a charging session at a specific charging point\\
	\hline
	WP4 & User disconnects the Electric Vehicle from the Charging Slot\\
	\hline
	WP5 & CPO starts the maintenance of a Charging Slot\\
	\hline
	WP6 & CPO completes the maintenance of a Charging Slot\\
	\hline
	WP7 & CPO decides to check the internal status of a charging station\\
	\hline
	WP8 & CPO decides to set up an energy management policy of one charging point\\
	\hline
	WP9 & CPO decides to change the price of tariffs\\
	\hline
	WP10 & CPO decides to change from which DSO acquire energy\\
	\hline
	WP11 & DSO provide energy to the charging point\\
	\hline
\end{tabular}


\subsection{Shared Phenomena}
\subsubsection{eMSPs side}
\begin{tabular}{|l|l|}
	\hline
	SP1 & eMSPs request information about the charging points\\
	\hline
	SP2 & eMSPs books a charging session \\
	\hline
	SP3 & eMSPs receives information about the state of the charging session\\
	\hline
\end{tabular}

\subsubsection{Charging point side}
\begin{tabular}{|l|l|}
	\hline
	SP4 & System authenticate the charging session\\
	\hline
	SP5 & The system activates a charging socket of an charging slot, making it available \\
	\hline
	SP6 & The system deactivates a charging socket of an charging slot\\
	\hline
	SP7 & charging slot sends to the system information about the charging status of the car\\
	\hline
	SP8 & charging slot communicates diagnostic information to the system\\
	\hline
\end{tabular}

\subsubsection{CPO side}
\begin{tabular}{|l|l|}
	\hline
	SP9 & The system shows the current energy management settings to the CPO\\
	\hline
	SP10 & CPO login into the system and change the energy management settings\\
	\hline
	SP13 & CPO login into the system and change the energy price\\
	\hline
	SP14 & CPO login into the system and view information about the internal status of the system\\
	\hline
	SP14 & System notifies the CPO of a component malfunction\\
	\hline
\end{tabular}


\section{Definitions, Acronyms, Abbreviations}

\subsection{Definitions}

\begin{tabular}{|p{5cm}|p{10cm}|}
	\hline
	Connector \newline Charging Socket & Physical connector that allow to transfer energy to the connected vehicle\\
	\hline
	Charging slot & Physical device with multiple Connectors that can charge electric vehicles.\newline
	NOTE: from OCPI definitions a Charging Slot can have up to one connector active at any time (i.e. can charge only one Vehicle at any time) \\
	\hline
	Charging Point & Physical structure composed by multiple Charging Slots\\
	\hline
	Maintenance of a charging slot & Activity/activities that results in a momentary unavailability of the charging slot\\
	\hline
	Charging session & period of time when the vehicle is connected to a charging plug for charging\\
	\hline
	Booking period & period of time between the booking of a charging session and the beginning of the charging session\\
	\hline
	Internal status & "The amount of energy stored in the batteries, if any, the quantity of vehicles currently being charged, and the current energy management settings. \\
	\hline
	
	
\end{tabular}

\subsection{Acronyms}
\begin{tabular}{|l|l|}
	\hline
	eMSP & e-Mobility Service Provider\\
	\hline
	CP & Charge Point / Charging Point\\
	\hline
	CS & Charge Slot / EVSE\\
	\hline
	CPO & Charging Point Operator\\
	\hline
	CPMS & Charging Point Management System\\
	\hline
	OCPI & Open Charge Point Interface\\
	\hline
	EV & Electric Vehicle\\
	\hline
	OCPP & Open Charge Point Protocol\\
	\hline
	DSO & Distribution Management System \\
	\hline
\end{tabular}

\section{Revision History}
\begin{itemize}
	\item v1.0 - 10 December 2022
	\item v1.1 - 23 December 2022 | fix naming 
\end{itemize}

\section{Related Documents}
\href{../Specs/OCPI-2.2.1.pdf}{OCPI specifications document}\\
\href{../Specs/ocpp-1.6.pdf.pdf}{OCPP specifications document}

\section{Document structure}
The document is structured in six sections:

\begin{enumerate}
	\item Problem introduction, associated goals of the project. In this section you can also find the scope of the project, the various phenomena occurring and the definitions and abbreviations used in this document.
	\item The second section contains a overview of the system, the details about the main functionalities. Class diagrams, state-charts and domain assumptions are used to introduce the various scenarios.
	\item In this section are specified the different requirements of the system, the various use cases and the mapping between functional requirements and the goals of the system.
	\item Alloy is used to conduct the formal analysis of the system.
	\item Total effort
	\item References used
\end{enumerate}
